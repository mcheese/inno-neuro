\section{Motiv}

\begin{slide}{Von Neumann Stärken}
	Stärken der von Neumann Architektur sind
	\begin{itemize}
		\item Universeller Einsatz
		\item Synchrone / Getaktete Verarbeitung
		\item Logik / Algorithmik
		\item Geschwindigkeit
	\end{itemize}
\end{slide}

\begin{slide}{Von Neumann Schwächen}
	Die von Neumann Architektur zeigt in bestimmten Einsatzgebieten schlechte Performanz.
	
	Schwierige Einsatzgebiete bei zeitkritischer Betrachtung sind bspw.
	\begin{itemize}
		\item Verarbeitung hochdimensionaler (verrauschter) Daten
		\item Gesichts- und Spracherkennung / Allg. Mustererkennung
		\item Sensoren-Netzwerke
	\end{itemize}
	
	Für die Verarbeitung in \alert{Echtzeit} ist je nach Anwendungsgebiet zum Teil ein sehr hoher Energieaufwand nötig.
\end{slide}

\begin{slide}{Motiv KNN Nutzung}
	Der Einsatz von \textbf{Künstlich Neuronalen Netzen} ist in den vorgestellten Problemgebieten zum Teil sehr erfolgreich. Ihre Simulation unter der von Neumann Architektur verhindert jedoch teilweise theoretische Vorteile.
\end{slide}

\begin{slide}{Motiv Neuro Architektur}
	Durch Entwicklung einer geeigneten Architektur für KNN erhofft man sich
	
	\begin{itemize}
		\item Hoch skalierbare Netzwerke
		\item Massive Parallelität
		\item Sehr geringen Energieaufwand
		\item Fehlertoleranz
		\item Geschwindigkeit
	\end{itemize}
	
	Vorbild dafür sind die Fähigkeiten des menschlichen Gehirns bei gerade einmal 20W Energieverbrauch.
\end{slide}

\begin{slide}{Profitierende Anwendungsgebiete}
	Von der Entwicklung von Neuro Chips profitieren vor allem zeitkritische Anwendungsgebiete, die bereits jetzt (theoretisch) von KNN profitieren.
	
	\begin{itemize}
		\item Autonomes Fahren
		\item Echtzeit Video Analyse
		\item Sensoren Netzwerke
	\end{itemize}
\end{slide}